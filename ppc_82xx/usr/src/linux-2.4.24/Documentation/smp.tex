From: michael@Physik.Uni-Dortmund.DE (Michael Dirkmann)

thanks for your information. Attached is the tex-code of your
SMP-documentation :
-=-=-=-=-=-=-=-=-=-=-=-=-=-=-=-=-=-=-=-=-=-=-=-=-=-=-=-=-=-=-=-=-=-=-=-=
\documentclass[]{article}
\parindent0.0cm
\parskip0.2cm

\begin{document}

\begin{center}
\LARGE \bf
An Implementation Of Multiprocessor Linux
\normalsize
\end{center}

{ \it
This document describes the implementation of a simple SMP 
Linux kernel extension and how to use this to develop SMP Linux kernels for 
architectures other than the Intel MP v1.1 architecture for Pentium and 486 
processors.}

\hfill Alan Cox, 1995


The author wishes to thank Caldera Inc. ( http://www.caldera.com )
whose donation of an ASUS dual Pentium board made this project possible, 
and Thomas Radke, whose initial work on multiprocessor Linux formed 
the backbone of this project.

\section{Background: The Intel MP specification.}
Most IBM PC style multiprocessor motherboards combine Intel 486 or Pentium 
processors and glue chipsets with a hardware/software specification. The 
specification places much of the onus for hard work on the chipset and 
hardware rather than the operating system.

The Intel Pentium processors have a wide variety of inbuilt facilities for 
supporting multiprocessing, including hardware cache coherency, built in 
interprocessor interrupt handling and a set of atomic test and set, 
exchange and similar operations. The cache coherency in particular makes the 
operating system's job far easier.

The specification defines a detailed configuration structure in ROM that 
the boot up processor can read to find the full configuration of the 
processors and buses. It also defines a procedure for starting up the 
other processors.


\section{Mutual Exclusion Within A Single Processor Linux Kernel}
For any kernel to function in a sane manner it has to provide internal 
locking and protection of its own tables to prevent two processes updating 
them at once and for example allocating the same memory block. There are 
two strategies for this within current Unix and Unixlike kernels. 
Traditional Unix systems from the earliest of days use a scheme of 'Coarse 
Grained Locking' where the entire kernel is protected by a small number of 
locks only. Some modern systems use fine grained locking. Because fine 
grained locking has more overhead it is normally used only on 
multiprocessor kernels and real time kernels. In a real time kernel the 
fine grained locking reduces the amount of time locks are held and reduces 
the critical (to real time programming at least) latency times.

Within the Linux kernel certain guarantees are made. No process running in 
kernel mode will be pre-empted by another kernel mode process unless it 
voluntarily sleeps.  This ensures that blocks of kernel code are 
effectively atomic with respect to other processes and greatly simplifies 
many operations. Secondly interrupts may pre-empt a kernel running process, 
but will always return to that process. A process in kernel mode may 
disable interrupts on the processor and guarantee such an interruption will 
not occur. The final guarantee is that an interrupt will not be pre-empted 
by a kernel task. That is interrupts will run to completion or be 
pre-empted by other interrupts only.

The SMP kernel chooses to continue these basic guarantees in order to make 
initial implementation and deployment easier.  A single lock is maintained 
across all processors. This lock is required to access the kernel space. 
Any processor may hold it and once it is held may also re-enter the kernel 
for interrupts and other services whenever it likes until the lock is 
relinquished. This lock ensures that a kernel mode process will not be 
pre-empted and ensures that blocking interrupts in kernel mode behaves 
correctly. This is guaranteed because only the processor holding the lock 
can be in kernel mode, only kernel mode processes can disable interrupts 
and only the processor holding the lock may handle an interrupt.

Such a choice is however poor for performance. In the longer term it is 
necessary to move to finer grained parallelism in order to get the best 
system performance. This can be done hierarchically by gradually refining 
the locks to cover smaller areas. With the current kernel highly CPU bound 
process sets perform well but I/O bound task sets can easily degenerate to 
near single processor performance levels. This refinement will be needed to 
get the best from Linux/SMP.

\subsection{Changes To The Portable Kernel Components}
The kernel changes are split into generic SMP support changes and 
architecture specific changes necessary to accommodate each different 
processor type Linux is ported to.


\subsubsection{Initialisation}
The first problem with a multiprocessor kernel is starting the other 
processors up. Linux/SMP defines that a single processor enters the normal 
kernel entry point start\_kernel(). Other processors are assumed not to be 
started or to have been captured elsewhere. The first processor begins the 
normal Linux initialisation sequences and sets up paging, interrupts and 
trap handlers. After it has obtained the processor information about the 
boot CPU, the architecture specific function 


{\tt \bf{void smp\_store\_cpu\_info(int processor\_id) }}

is called to store any information about the processor into a per processor 
array. This includes things like the bogomips speed ratings.

Having completed the kernel initialisation the architecture specific 
function

{\tt \bf void smp\_boot\_cpus(void) }

is called and is expected to start up each other processor and cause it to 
enter start\_kernel() with its paging registers and other control 
information correctly loaded. Each other processor skips the setup except 
for calling the trap and irq initialisation functions that are needed on 
some processors to set each CPU up correctly.  These functions will 
probably need to be modified in existing kernels to cope with this.


Each additional CPU then calls the architecture specific function

{\tt \bf void smp\_callin(void)}

which does any final setup and then spins the processor while the boot 
up processor forks off enough idle threads for each processor. This is 
necessary because the scheduler assumes there is always something to run. 
Having generated these threads and forked init the architecture specific 

{\tt \bf void smp\_commence(void)}

function is invoked. This does any final setup and indicates to the system 
that multiprocessor mode is now active. All the processors spinning in the 
smp\_callin() function are now released to run the idle processes, which 
they will run when they have no real work to process.


\subsubsection{Scheduling}
The kernel scheduler implements a simple but very effective task 
scheduler. The basic structure of this scheduler is unchanged in the 
multiprocessor kernel. A processor field is added to each task, and this 
maintains the number of the processor executing a given task, or a magic 
constant (NO\_PROC\_ID)  indicating the job is not allocated to a processor. 
	 
Each processor executes the scheduler itself and will select the next task 
to run from all runnable processes not allocated to a different processor. 
The algorithm used by the selection is otherwise unchanged. This is 
actually inadequate for the final system because there are advantages to 
keeping a process on the same CPU, especially on processor boards with per 
processor second level caches.

Throughout the kernel the variable 'current' is used as a global for the 
current process. In Linux/SMP this becomes a macro which expands to 
current\_set[smp\_processor\_id()]. This enables almost the entire kernel to 
be unaware of the array of running processors, but still allows the SMP 
aware kernel modules to see all of the running processes.

The fork system call is modified to generate multiple processes with a 
process id of zero until the SMP kernel starts up properly. This is 
necessary because process number 1 must be init, and it is desirable that 
all the system threads are process 0. 

The final area within the scheduling of processes that does cause problems 
is the fact the uniprocessor kernel hard codes tests for the idle threads 
as task[0] and the init process as task[1]. Because there are multiple idle 
threads it is necessary to replace these with tests that the process id is 
0 and a search for process ID 1, respectively.

\subsubsection{Memory Management}
The memory management core of the existing Linux system functions 
adequately within the multiprocessor framework providing the locking is 
used. Certain processor specific areas do need changing, in particular 
invalidate() must invalidate the TLBs of all processors before it returns.


\subsubsection{Miscellaneous Functions}
The portable SMP code rests on a small set of functions and variables 
that are provided by the processor specification functionality. These are

{\tt \bf int smp\_processor\_id(void) }

which returns the identity of the processor the call is executed upon. This 
call is assumed to be valid at all times. This may mean additional tests 
are needed during initialisation.


{\tt \bf int smp\_num\_cpus;}

This is the number of processors in the system. \

{\tt \bf void smp\_message\_pass(int target, int msg, unsigned long data,
		int wait)}

This function passes messages between processors. At the moment it is not 
sufficiently defined to sensibly document and needs cleaning up and further 
work. Refer to the processor specific code documentation for more details.


\subsection{Architecture Specific Code For the Intel MP Port}
The architecture specific code for the Intel port splits fairly cleanly 
into four sections. Firstly the initialisation code used to boot the 
system, secondly the message handling and support code, thirdly the 
interrupt and kernel syscall entry function handling and finally the 
extensions to standard kernel facilities to cope with multiple processors.

\subsubsection{Initialisation}	
The Intel MP architecture captures all the processors except for a single 
processor known as the 'boot processor' in the BIOS at boot time. Thus a 
single processor enters the kernel bootup code. The first processor 
executes the bootstrap code, loads and uncompresses the kernel. Having 
unpacked the kernel it sets up the paging and control registers then enters 
the C kernel startup.

The assembler startup code for the kernel is modified so that it can be 
used by the other processors to do the processor identification and various 
other low level configurations but does not execute those parts of the 
startup code that would damage the running system (such as clearing the BSS 
segment). 

In the initialisation done by the first processor the arch/i386/mm/init 
code is modified to scan the low page, top page and BIOS for intel MP 
signature blocks. This is necessary because the MP signature blocks must 
be read and processed before the kernel is allowed to allocate and destroy 
the page at the top of low memory. Having established the number of 
processors it reserves a set of pages to provide a stack come boot up area 
for each processor in the system. These must be allocated at startup to 
ensure they fall below the 1Mb boundary.

Further processors are started up in smp\_boot\_cpus() by programming the 
APIC controller registers and sending an inter-processor interrupt (IPI) to 
the processor. This message causes the target processor to begin executing 
code at the start of any page of memory within the lowest 1Mb, in 16bit 
real mode. The kernel uses the single page it allocated for each processor 
to use as stack. Before booting a given CPU the relocatable code from 
trampoline.S and trampoline32.S is copied to the bottom of its stack page 
and used as the target for the startup. 

The trampoline code calculates the desired stack base from the code 
segment (since the code segment on startup is the bottom of the stack), 
 enters 32bit mode and jumps to the kernel entry assembler. This as 
described above is modified to only execute the parts necessary for each 
processor, and then to enter start\_kernel(). On entering the kernel the 
processor initialises its trap and interrupt handlers before entering 
smp\_callin(), where it reports its status and sets a flag that causes the 
boot processor to continue and look for further processors. The processor 
then spins until smp\_commence() is invoked.

Having started each processor up the smp\_commence( ) function flips a 
flag. Each processor spinning in smp\_callin() then loads the task register 
with the task state segment (TSS) of its idle thread as is needed for task 
switching.

\subsubsection{Message Handling and Support Code}
The architecture specific code implements the smp\_processor\_id() function 
by querying the APIC logical identity register. Because the APIC isn't 
mapped into the kernel address space at boot, the initial value returned is 
rigged by setting the APIC base pointer to point at a suitable constant. 
Once the system starts doing the SMP setup (in smp\_boot\_cpus()), the APIC 
is mapped with a vremap() call and the apic pointer is adjusted 
appropriately. From then on the real APIC logical identity register is 
read.

Message passing is accomplished using a pair of IPIs on interrupt 13 
(unused by the 80486 FPUs in SMP mode) and interrupt 16. Two are used in 
order to separate messages that cannot be processed until the receiver 
obtains the kernel spinlock from messages that can be processed 
immediately. In effect IRQ 13 is a fast IRQ handler that does not obtain 
the locks, and cannot cause a reschedule, while IRQ 16 is a slow IRQ that 
must acquire the kernel spinlocks and can cause a reschedule. This 
interrupt is used for passing on slave timer messages from the processor 
that receives the timer interrupt to the rest of the processors, so that 
they can reschedule running tasks.


\subsubsection{Entry And Exit Code}
A single spinlock protects the entire kernel. The interrupt handlers, the 
syscall entry code and the exception handlers all acquire the lock before 
entering the kernel proper. When the processor is trying to acquire the 
spinlock it spins continually on the lock with interrupts disabled. This 
causes a specific deadlock problem. The lock owner may need to send an 
invalidate request to the rest of the processors and wait for these to 
complete before continuing. A processor spinning on the lock would not be 
able to do this. Thus the loop of the spinlock tests and handles invalidate 
requests. If the invalidate bit for the spinning CPU is set the processor 
invalidates its TLB and atomically clears the bit. When the spinlock is 
obtained that processor will take an IPI and in the IPI test the bit and 
skip the invalidate as the bit is clear.

One complexity of the spinlock is that a process running in kernel mode 
can sleep voluntarily and be pre-empted. A switch from such a process to a 
process executing in user space may reduce the lock count. To track this 
the kernel uses a syscall\_count and a per process lock\_depth parameter to 
track the kernel lock state. The switch\_to() function is modified in SMP 
mode to adjust the lock appropriately.

The final problem is the idle thread. In the single processor kernel the 
idle thread executes 'hlt' instructions. This saves power and reduces the 
running temperature of the processors when they are idle. However it means 
the process spends all its time in kernel mode and would thus hold the 
kernel spinlock. The SMP idle thread continually reschedules a new task and 
returns to user mode. This is far from ideal and will be modified to use 
'hlt' instructions and release the spinlock soon. Using 'hlt' is even more 
beneficial on a multiprocessor system as it almost completely takes an idle 
processor off the bus.

Interrupts are distributed by an i82489 APIC. This chip is set up to work 
as an emulation of the traditional PC interrupt controllers when the 
machine boots (so that an Intel MP machine boots one CPU and PC 
compatible). The kernel has all the relevant locks but does not yet 
reprogram the 82489 to deliver interrupts to arbitrary processors as it 
should. This requires further modification of the standard Linux interrupt 
handling code, and is particularly messy as the interrupt handler behaviour 
has to change as soon as the 82489 is switched into SMP mode.


\subsubsection{Extensions To Standard Facilities}
The kernel maintains a set of per processor control information such as 
the speed of the processor for delay loops. These functions on the SMP 
kernel look the values up in a per processor array that is set up from the 
data generated at boot up by the smp\_store\_cpu\_info() function. This 
includes other facts such as whether there is an FPU on the processor. The 
current kernel does not handle floating point correctly, this requires some 
changes to the techniques the single CPU kernel uses to minimise floating 
point processor reloads.

The highly useful atomic bit operations are prefixed with the 'lock' 
prefix in the SMP kernel to maintain their atomic properties when used 
outside of (and by) the spinlock and message code. Amongst other things 
this is needed for the invalidate handler, as all  CPU's will invalidate at 
the same time without any locks.

Interrupt 13 floating point error reporting is removed. This facility is 
not usable on a multiprocessor board, nor relevant to the Intel MP 
architecture which does not cover the 80386/80387 processor pair. \

The /proc filesystem support is changed so that the /proc/cpuinfo file 
contains a column for each processor present. This information is extracted 
from the data saved by smp\_store\_cpu\_info().

\end{document}
